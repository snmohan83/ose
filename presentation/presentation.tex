%===============================================================================
% Zweck:    OSE-Präsentation-Vorlage
% Erstellt: 15.04.2013
% Update:   04.07.2016
% Autor:    M.G.
%===============================================================================
\RequirePackage[hyphens]{url}
\newcommand\ratio{169}
\documentclass[10pt,aspectratio=\ratio,
%draft,
%handout,
compress
]{beamer}

\newcommand\meta{./meta}
\input{\meta/config/commands}


\def\signed #1{{\leavevmode\unskip\nobreak\hfil\penalty50\hskip2em
  \hbox{}\nobreak\hfil(#1)%
  \parfillskip=0pt \finalhyphendemerits=0 \endgraf}}

\newsavebox\mybox
\newenvironment{aquote}[1]
  {\savebox\mybox{#1}\begin{fancyquotes}}
  {\signed{\usebox\mybox}\end{fancyquotes}}


\input{\meta/config/hyphenation}

\setbeamertemplate{caption}[numbered]
%\numberwithin{figure}{section}
\usepackage{listings}
\usepackage{parcolumns}
\begin{document}
  %===============================================================================
  % Zum Kompilieren latexmk ausführen.
  %	Konfiguration in texmaker: Options -> Configure Texmaker -> Quick Build -> Select Latexmk + ViewPD
  %	Entsprechende Informationen in den config/metainfo verändern
  % Zur Auswahl der Sprache im folgenden Befehl
  % ngerman für deutsch eintragen, english für Englisch.
  %===============================================================================
\selectlanguage{english}
\ifnum\ratio<169
\frame{\titlepage}
\else
\frame[plain]{\titlepage}
\fi

%\AtBeginSection[]
%{
%  \frame<handout:0>
%  {
%    \frametitle{Outline}
%    \tableofcontents[currentsection,hideallsubsections]
%  }
%}

\AtBeginSubsection[]
{
  \frame<handout:0>
  {
    \frametitle{Outline}
    \tableofcontents[sectionstyle=show/hide,subsectionstyle=show/shaded/hide,subsubsectionstyle=hide]
  }
}

\AtBeginSubsubsection[]
{
  \frame<handout:0>
  {
    \frametitle{Outline}
    \tableofcontents[sectionstyle=show/hide,subsectionstyle=show/shaded/hide,subsubsectionstyle=show/shaded/hide]
  }
}

\newcommand<>{\highlighton}[1]{%
  \alt#2{\structure{#1}}{{#1}}
}

\newcommand{\icon}[1]{\pgfimage[height=1em]{#1}}

\section*{}
\phantomsection
\begin{frame}{Content}
\tableofcontents
\end{frame}
%%%%%%%%%%%%%%%%%%%%%%%%%%%%%%%%%%%%%%%%%
%%%%%%%%%% Content starts here %%%%%%%%%%
%%%%%%%%%%%%%%%%%%%%%%%%%%%%%%%%%%%%%%%%%

\section{Introduction}
\begin{frame}{Introduction}
    \begin{itemize}                                                                             
        \item The Synthesis Kernel \cite{synthesis} introduces runtime code generation as one of the key ideas for an efficient operating system.
        \item Runtime Code generation enables optimization
        \item \cite{synthesis} proposes common optimization techniques like - constant folding, constant propagation and procedure inlining
        \item The Kernel and the subsequent paper on Superoptimizer by Henry Massalin tend to achieve the goal of aiding the expert to optimize better.
    \end{itemize}
\end{frame}

\section{Learnings from Synthesis OS Kernel}
\begin{frame}{Synthesis OS Kernel} 
    \begin{itemize}
        \item Synthesis optimized code for frequently used Kernel routines - queues, buffers, switchers, interrupt handlers, system call dispatchers
        \item Fine-grained scheduling - Deduce CPU time for each task through measurement and data accumulation
        \item Tackling the goal of high throughput with low latency
        \item Optimization through design changes in various functions of the kernel - Diffuse Kernel, I/O, process management, services and interfaces
    \end{itemize}
\end{frame}


\begin{frame}{Optimizations from Synthesis OS}
    \begin{itemize}
        \item Constant Folding, Constant Propagation and Procedure Inlining are used
        \item Data-aware optimizations using information available at compile-time
        \item TradeOff: Cost saving must exceed code generation cost
        \item Methods used by Synthesis for code generation:
            \begin{itemize}
                \item Facoring invariants - partial evaluation for constraints
                \item Collapsing Layers - similar to inlining reaching multiple layers
                \item Executable Data structures - self-traversing - Jobqueue with Startjob and Stopjob sequences
            \end{itemize}
        \item Challenges: Size, Protection and Cache Coherence
    \end{itemize}
\end{frame}


\section{Code-Level Optimizations}

\begin{frame}{Traditional Optimization Tricks}
Earlier processors had constraints of memory and CPU cycles and every cycle and pipeline stalls mattered. 

Imagine running a video decoding application on a 16 KB RAM and 128 Mhz CPU.
\bigskip
    \begin{enumerate}
        \item Replace decision constructs with logical and arithmetic operations
        \item Utilize processor's parallel processing capabilities avoiding pipeline stalls
          \begin{itemize}
            \item Unroll loops to avoid pipeline dependencies
            \item Algorithmic changes to parallelize Load/Store with MAC instructions      
          \end{itemize}
        \item Precompute constants and constant expressions
        \item Inline smaller frequently called functions/procedures 
    \end{enumerate}
    The last two ideas were also mentioned in \cite{synthesis}
\end{frame}

\begin{frame}[fragile]{Some Examples of Code-level Optimizations}
  \begin{lstlisting}[backgroundcolor=\color{lightgray}]
    int maximum(int a, int b) {
        if (a > b)
            return a;
        else
            return b;
    }
  \end{lstlisting}
  Can be replaced by:
  \begin{lstlisting}[backgroundcolor=\color{lightgray}]
    int maximum(int a, int b) {
        return -(((b - a) >> 31)*a + ((a - b) >> 31)*b);
    }
  \end{lstlisting}
\end{frame}

\begin{frame}{Pipeline Stages and Loop Unrolling}
This is a sample set of processor pipeline stages of a RISC processor and can be different for other CPUs.
Assumption: Each instruction takes 1 cycle unless otherwise mentioned.
\image{0.8\textwidth}{\meta/config/images/fivestagespipeline.png}{Pipeline Stages RISC from Wikipedia on RISC Pipeline \cite{WikipediaEN:RISC} }{img: fivestagespipeline}
\end{frame}

\begin{frame}[fragile]{Loop Unrolling Example}
\noindent\begin{minipage}{.45\textwidth}
  \begin{lstlisting}[backgroundcolor=\color{lightgray}]
    Loop:
      load R1, (R6)
      load R2, (R6+1)
      mac R1, R2, R3  ..2ticks
      store R3, (R9)
      add R6, 2
      inc R7 
    EndLoop
  \end{lstlisting}
\end{minipage} \hfill
\begin{minipage}{.45\textwidth}
  \begin{lstlisting}[basicstyle=\small, backgroundcolor=\color{lightgray}]
    load R1, (R6)
    load R2, (R6+1)
    add R6, 2
    Loop:  
      Load R4, (R6)
      mac R1, R2, R3 ..2ticks
      Load R5, (R6+1)
    
      add R6, 2
      store R3, (R9)
    
      Load R1,(R6)
      mac R4, R5, R7
      Load R2, (R6+1)
    
      add R6, 2
      store R7, (R9)
      inc R9
    Endloop
  \end{lstlisting}
\end{minipage}
\end{frame}


\section{SuperOptimizers}

\begin{frame}{Superoptimizer Methods}
Goal: Given an instruction set, search for the smallest program for a task. Introduced first by Henrz Massalin \cite{superopt}.
\image{0.8\textwidth}{\meta/config/images/superoptimizer.png}{Superoptimizer Methods}{img: superoptimizer}
\end{frame}

\begin{frame}{Limitations and Applications}
  \begin{itemize}
    \item Exponential growth of search time with instruction set (several hours)
    \item Modeling a pointer is difficult (take whole memory into account)
    \item Machine-independence
  \end{itemize}
  \begin{itemize}
    \item Effective in register-register operations
    \item Design of RISC architectures
    \item For program snippets overlooked by compiler (eg> multiplication by constants)
    \item Best used to aid assembly language programmers
  \end{itemize}
\end{frame}

\section{Peephole SuperOptimizers}
\begin{frame}{Automatic Generation of Peephole Superoptimizers}
    \begin{itemize}
        \item Peephole optimizers replace sequence of instructions with faster sequence
        \item Utilizes pattern-matching to decide rules - Automation to use superoptimization
        \item implemented as a network-based search engine on a database of learned optimizations
    \end{itemize}
\image{0.8\textwidth}{\meta/config/images/autosuperopt.png}{Architecture of automated SuperOptimizer \cite{PeepholeAutogen}}{img: autosuperopt}
\end{frame}


\section{SuperOptimizers in Modern Compilers}
\begin{frame}[fragile]{GNU Superoptimizer}
    \begin{itemize}
        \item Gives the shortest instruction sequence
        \item Available as an installable package 
        \item Supports multiple CPU Architectures
        \item SPARC, MC68000, MC68020, M88000, POWER, POWERPC, AM29K, I386, I960 (for i960 1.0), I960B (for I960B 1.1), PYR, ALPHA, HPPA, SH
    \end{itemize}
    To compile and run the superopt on a GNU Machine
    \begin{lstlisting}[backgroundcolor=\color{lightgray}]
        > make CPU=-D<cpu_name_from_list> superopt
        >
        > superopt -f<goal-function> | -all  [-assembly] 
        [-max-cost n] [-shifts] [-extracts] [-no-carry-insns] 
        [-extra-cost n]
    \end{lstlisting}
\end{frame}


\section{Conclusions and Future Work}
\begin{frame}{Conclusions}
    \begin{itemize}
        \item Optimized programs are always beneficial particularly so in low-level systems
        \item Ideas in Synthesis are relevant and have been used and improved since
        \item Superoptimization can provide great performance benefits
        \item Limitations on size, effort, and time can be countered through automation
        \item Modern compilers try to incorporate some of the techniques
    \end{itemize}
    Future Work:
    \begin{itemize}
        \item Improvements have been made on the superoptimization originally proposed in \cite{superopt}
        \item Goal-oriented, automated and efficient superoptimization techniques have evolved over the years
        \item Standard compilers also have tools to aid superoptimization
        \item Further research is required into reducing the search space, optimal equivalence test and possible application of machine learning techniques to train superoptimizers can be thought through.
    \end{itemize}
\end{frame}


%%%%%%%%%%%%%%%%%%%%%%%%%%%%%%%%%%%%%%%%%
%%%%%%%%%% References          %%%%%%%%%%
%%%%%%%%%%%%%%%%%%%%%%%%%%%%%%%%%%%%%%%%%
\section*{}
\begin{frame}[allowframebreaks]{References}
 \def\newblock{\hskip .11em plus .33em minus .07em}
 \scriptsize
 \bibliographystyle{IEEEtran}
 \bibliography{\meta/exampleLiterature/bib}
 \normalsize
\end{frame}

%% Last frame
\frame{
  \vspace{2cm}
  {\huge Questions ?}

  \vspace{20mm}
  \nocite*

  \begin{flushright}
    Madhu Mohan Nelemane

    \structure{\footnotesize{\href{mailto:sridhara-madhu-mohan.nelemane@stud.uni-bamberg.de}{sridhara-madhu-mohan.nelemane@stud.uni-bamberg.de}}}
  \end{flushright}
}


\end{document}
