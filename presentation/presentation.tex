%===============================================================================
% Zweck:    KTR-Präsentation-Vorlage
% Erstellt: 15.04.2013
% Update:   04.07.2016
% Autor:    M.G.
%===============================================================================
\RequirePackage[hyphens]{url}
\newcommand\ratio{169}
\documentclass[10pt,aspectratio=\ratio,
%draft,
%handout,
compress
]{beamer}

\newcommand\meta{./meta}
\input{\meta/config/commands}


\def\signed #1{{\leavevmode\unskip\nobreak\hfil\penalty50\hskip2em
  \hbox{}\nobreak\hfil(#1)%
  \parfillskip=0pt \finalhyphendemerits=0 \endgraf}}

\newsavebox\mybox
\newenvironment{aquote}[1]
  {\savebox\mybox{#1}\begin{fancyquotes}}
  {\signed{\usebox\mybox}\end{fancyquotes}}



\input{\meta/config/hyphenation}

\setbeamertemplate{caption}[numbered]
%\numberwithin{figure}{section}

\begin{document}
  %===============================================================================
  % Zum Kompilieren latexmk ausführen.
  %	Konfiguration in texmaker: Options -> Configure Texmaker -> Quick Build -> Select Latexmk + ViewPD
  %	Entsprechende Informationen in den config/metainfo verändern
  % Zur Auswahl der Sprache im folgenden Befehl
  % ngerman für deutsch eintragen, english für Englisch.
  %===============================================================================
\selectlanguage{english}
\ifnum\ratio<169
\frame{\titlepage}
\else
\frame[plain]{\titlepage}
\fi

%\AtBeginSection[]
%{
%  \frame<handout:0>
%  {
%    \frametitle{Outline}
%    \tableofcontents[currentsection,hideallsubsections]
%  }
%}

\AtBeginSubsection[]
{
  \frame<handout:0>
  {
    \frametitle{Outline}
    \tableofcontents[sectionstyle=show/hide,subsectionstyle=show/shaded/hide,subsubsectionstyle=hide]
  }
}

\AtBeginSubsubsection[]
{
  \frame<handout:0>
  {
    \frametitle{Outline}
    \tableofcontents[sectionstyle=show/hide,subsectionstyle=show/shaded/hide,subsubsectionstyle=show/shaded/hide]
  }
}

\newcommand<>{\highlighton}[1]{%
  \alt#2{\structure{#1}}{{#1}}
}

\newcommand{\icon}[1]{\pgfimage[height=1em]{#1}}

\section*{}
\phantomsection
\begin{frame}{Content}
\tableofcontents
\end{frame}
%%%%%%%%%%%%%%%%%%%%%%%%%%%%%%%%%%%%%%%%%
%%%%%%%%%% Content starts here %%%%%%%%%%
%%%%%%%%%%%%%%%%%%%%%%%%%%%%%%%%%%%%%%%%%


\section{Introduction}
\begin{frame}{Introduction}
This presentation talks about superoptimization
\end{frame}
\section{Simple Assembly language Optimization Techniques}
\begin{frame}{Assembly}
\begin{itemize}
\item Replace decision constructs with logical and arithmetic operations
\item Unroll loops to avoid pipeline dependencies
\item Parallelize Load/Store with MAC instructions
\end{itemize}
\end{frame}
\section{Learnings from Synthesis kernel and OS}
\begin{frame}{Optimizations in Synthesis Kernel}
\begin{itemize}
\item Dynamic Code Generation
\item Software Feedback
\end{itemize}
\end{frame}

\begin{frame}{Optimizations from Synthesis OS}

\end{frame}

\section{Introducing SuperOptimizers}
\begin{frame}{Basic Superoptimizer steps}
\begin{itemize}
\item Boolean Test
\item Probabilistic Test
\item Pruning
\item Applications and Limitations
\end{itemize}
\end{frame}

\section{Peephole Superoptimizers and Modern Compilers}
\begin{frame}{Automatic Generation of Peephole Superoptimizers}
\end{frame}
\begin{frame}{Improving Binary Translations using Peephole superoptimizers}
\end{frame}
\begin{frame}{GNU Superoptimizer}
\begin{itemize}
\item https://www.gnu.org/software/superopt/
\item Usage of superopt command, options and results
\end{itemize}
\end{frame}
\begin{frame}{LLVM and JIT}
\end{frame}
\section{Conclusion}
\begin{frame}{Conclusion}
\end{frame}


%%%%%%%%%%%%%%%%%%%%%%%%%%%%%%%%%%%%%%%%%
%%%%%%%%%% References          %%%%%%%%%%
%%%%%%%%%%%%%%%%%%%%%%%%%%%%%%%%%%%%%%%%%
\section*{}
\begin{frame}[allowframebreaks]{References}
 \def\newblock{\hskip .11em plus .33em minus .07em}
 \scriptsize
 \bibliographystyle{IEEEtran}
 \bibliography{\meta/exampleLiterature/bib}
 \normalsize
\end{frame}

%% Last frame
\frame{
  \vspace{2cm}
  {\huge Questions ?}

  \vspace{20mm}
  \nocite*

  \begin{flushright}
    Marcel Gro\ss mann

    \structure{\footnotesize{\href{mailto:marcel.grossmann@uni-bamberg.de}{marcel.grossmann@uni-bamberg.de}}}
  \end{flushright}
}


\end{document}
