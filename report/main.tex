\documentclass[sigconf, language=english]{acmart}

% if you need to pass options to natbib, use, e.g.:
%     \PassOptionsToPackage{numbers, compress}{natbib}
% before loading neurips_2022


% ready for submission
%\usepackage{ofu_xai_2022}

% SOURCE: https://github.com/goodfeli/dlbook_notation/blob/master/math_commands.tex
% Quote from github "We make them freely available for anyone to use."

\usepackage{amsmath,amsfonts,bm}


%%%%% NEW MATH DEFINITIONS %%%%%

% Mark sections of captions for referring to divisions of figures
\newcommand{\figleft}{{\em (Left)}}
\newcommand{\figcenter}{{\em (Center)}}
\newcommand{\figright}{{\em (Right)}}
\newcommand{\figtop}{{\em (Top)}}
\newcommand{\figbottom}{{\em (Bottom)}}
\newcommand{\captiona}{{\em (a)}}
\newcommand{\captionb}{{\em (b)}}
\newcommand{\captionc}{{\em (c)}}
\newcommand{\captiond}{{\em (d)}}

% Highlight a newly defined term
\newcommand{\newterm}[1]{{\bf #1}}


% Figure reference, lower-case.
\def\figref#1{figure~\ref{#1}}
% Figure reference, capital. For start of sentence
\def\Figref#1{Figure~\ref{#1}}
\def\twofigref#1#2{figures \ref{#1} and \ref{#2}}
\def\quadfigref#1#2#3#4{figures \ref{#1}, \ref{#2}, \ref{#3} and \ref{#4}}
% Section reference, lower-case.
\def\secref#1{section~\ref{#1}}
% Section reference, capital.
\def\Secref#1{Section~\ref{#1}}
% Reference to two sections.
\def\twosecrefs#1#2{sections \ref{#1} and \ref{#2}}
% Reference to three sections.
\def\secrefs#1#2#3{sections \ref{#1}, \ref{#2} and \ref{#3}}
% Reference to an equation, lower-case.
\def\eqref#1{equation~\ref{#1}}
% Reference to an equation, upper case
\def\Eqref#1{Equation~\ref{#1}}
% A raw reference to an equation---avoid using if possible
\def\plaineqref#1{\ref{#1}}
% Reference to a chapter, lower-case.
\def\chapref#1{chapter~\ref{#1}}
% Reference to an equation, upper case.
\def\Chapref#1{Chapter~\ref{#1}}
% Reference to a range of chapters
\def\rangechapref#1#2{chapters\ref{#1}--\ref{#2}}
% Reference to an algorithm, lower-case.
\def\algref#1{algorithm~\ref{#1}}
% Reference to an algorithm, upper case.
\def\Algref#1{Algorithm~\ref{#1}}
\def\twoalgref#1#2{algorithms \ref{#1} and \ref{#2}}
\def\Twoalgref#1#2{Algorithms \ref{#1} and \ref{#2}}
% Reference to a part, lower case
\def\partref#1{part~\ref{#1}}
% Reference to a part, upper case
\def\Partref#1{Part~\ref{#1}}
\def\twopartref#1#2{parts \ref{#1} and \ref{#2}}

\def\ceil#1{\lceil #1 \rceil}
\def\floor#1{\lfloor #1 \rfloor}
\def\1{\bm{1}}
\newcommand{\train}{\mathcal{D}}
\newcommand{\valid}{\mathcal{D_{\mathrm{valid}}}}
\newcommand{\test}{\mathcal{D_{\mathrm{test}}}}

\def\eps{{\epsilon}}


% Random variables
\def\reta{{\textnormal{$\eta$}}}
\def\ra{{\textnormal{a}}}
\def\rb{{\textnormal{b}}}
\def\rc{{\textnormal{c}}}
\def\rd{{\textnormal{d}}}
\def\re{{\textnormal{e}}}
\def\rf{{\textnormal{f}}}
\def\rg{{\textnormal{g}}}
\def\rh{{\textnormal{h}}}
\def\ri{{\textnormal{i}}}
\def\rj{{\textnormal{j}}}
\def\rk{{\textnormal{k}}}
\def\rl{{\textnormal{l}}}
% rm is already a command, just don't name any random variables m
\def\rn{{\textnormal{n}}}
\def\ro{{\textnormal{o}}}
\def\rp{{\textnormal{p}}}
\def\rq{{\textnormal{q}}}
\def\rr{{\textnormal{r}}}
\def\rs{{\textnormal{s}}}
\def\rt{{\textnormal{t}}}
\def\ru{{\textnormal{u}}}
\def\rv{{\textnormal{v}}}
\def\rw{{\textnormal{w}}}
\def\rx{{\textnormal{x}}}
\def\ry{{\textnormal{y}}}
\def\rz{{\textnormal{z}}}

% Random vectors
\def\rvepsilon{{\mathbf{\epsilon}}}
\def\rvtheta{{\mathbf{\theta}}}
\def\rva{{\mathbf{a}}}
\def\rvb{{\mathbf{b}}}
\def\rvc{{\mathbf{c}}}
\def\rvd{{\mathbf{d}}}
\def\rve{{\mathbf{e}}}
\def\rvf{{\mathbf{f}}}
\def\rvg{{\mathbf{g}}}
\def\rvh{{\mathbf{h}}}
\def\rvu{{\mathbf{i}}}
\def\rvj{{\mathbf{j}}}
\def\rvk{{\mathbf{k}}}
\def\rvl{{\mathbf{l}}}
\def\rvm{{\mathbf{m}}}
\def\rvn{{\mathbf{n}}}
\def\rvo{{\mathbf{o}}}
\def\rvp{{\mathbf{p}}}
\def\rvq{{\mathbf{q}}}
\def\rvr{{\mathbf{r}}}
\def\rvs{{\mathbf{s}}}
\def\rvt{{\mathbf{t}}}
\def\rvu{{\mathbf{u}}}
\def\rvv{{\mathbf{v}}}
\def\rvw{{\mathbf{w}}}
\def\rvx{{\mathbf{x}}}
\def\rvy{{\mathbf{y}}}
\def\rvz{{\mathbf{z}}}

% Elements of random vectors
\def\erva{{\textnormal{a}}}
\def\ervb{{\textnormal{b}}}
\def\ervc{{\textnormal{c}}}
\def\ervd{{\textnormal{d}}}
\def\erve{{\textnormal{e}}}
\def\ervf{{\textnormal{f}}}
\def\ervg{{\textnormal{g}}}
\def\ervh{{\textnormal{h}}}
\def\ervi{{\textnormal{i}}}
\def\ervj{{\textnormal{j}}}
\def\ervk{{\textnormal{k}}}
\def\ervl{{\textnormal{l}}}
\def\ervm{{\textnormal{m}}}
\def\ervn{{\textnormal{n}}}
\def\ervo{{\textnormal{o}}}
\def\ervp{{\textnormal{p}}}
\def\ervq{{\textnormal{q}}}
\def\ervr{{\textnormal{r}}}
\def\ervs{{\textnormal{s}}}
\def\ervt{{\textnormal{t}}}
\def\ervu{{\textnormal{u}}}
\def\ervv{{\textnormal{v}}}
\def\ervw{{\textnormal{w}}}
\def\ervx{{\textnormal{x}}}
\def\ervy{{\textnormal{y}}}
\def\ervz{{\textnormal{z}}}

% Random matrices
\def\rmA{{\mathbf{A}}}
\def\rmB{{\mathbf{B}}}
\def\rmC{{\mathbf{C}}}
\def\rmD{{\mathbf{D}}}
\def\rmE{{\mathbf{E}}}
\def\rmF{{\mathbf{F}}}
\def\rmG{{\mathbf{G}}}
\def\rmH{{\mathbf{H}}}
\def\rmI{{\mathbf{I}}}
\def\rmJ{{\mathbf{J}}}
\def\rmK{{\mathbf{K}}}
\def\rmL{{\mathbf{L}}}
\def\rmM{{\mathbf{M}}}
\def\rmN{{\mathbf{N}}}
\def\rmO{{\mathbf{O}}}
\def\rmP{{\mathbf{P}}}
\def\rmQ{{\mathbf{Q}}}
\def\rmR{{\mathbf{R}}}
\def\rmS{{\mathbf{S}}}
\def\rmT{{\mathbf{T}}}
\def\rmU{{\mathbf{U}}}
\def\rmV{{\mathbf{V}}}
\def\rmW{{\mathbf{W}}}
\def\rmX{{\mathbf{X}}}
\def\rmY{{\mathbf{Y}}}
\def\rmZ{{\mathbf{Z}}}

% Elements of random matrices
\def\ermA{{\textnormal{A}}}
\def\ermB{{\textnormal{B}}}
\def\ermC{{\textnormal{C}}}
\def\ermD{{\textnormal{D}}}
\def\ermE{{\textnormal{E}}}
\def\ermF{{\textnormal{F}}}
\def\ermG{{\textnormal{G}}}
\def\ermH{{\textnormal{H}}}
\def\ermI{{\textnormal{I}}}
\def\ermJ{{\textnormal{J}}}
\def\ermK{{\textnormal{K}}}
\def\ermL{{\textnormal{L}}}
\def\ermM{{\textnormal{M}}}
\def\ermN{{\textnormal{N}}}
\def\ermO{{\textnormal{O}}}
\def\ermP{{\textnormal{P}}}
\def\ermQ{{\textnormal{Q}}}
\def\ermR{{\textnormal{R}}}
\def\ermS{{\textnormal{S}}}
\def\ermT{{\textnormal{T}}}
\def\ermU{{\textnormal{U}}}
\def\ermV{{\textnormal{V}}}
\def\ermW{{\textnormal{W}}}
\def\ermX{{\textnormal{X}}}
\def\ermY{{\textnormal{Y}}}
\def\ermZ{{\textnormal{Z}}}

% Vectors
\def\vzero{{\bm{0}}}
\def\vone{{\bm{1}}}
\def\vmu{{\bm{\mu}}}
\def\vtheta{{\bm{\theta}}}
\def\va{{\bm{a}}}
\def\vb{{\bm{b}}}
\def\vc{{\bm{c}}}
\def\vd{{\bm{d}}}
\def\ve{{\bm{e}}}
\def\vf{{\bm{f}}}
\def\vg{{\bm{g}}}
\def\vh{{\bm{h}}}
\def\vi{{\bm{i}}}
\def\vj{{\bm{j}}}
\def\vk{{\bm{k}}}
\def\vl{{\bm{l}}}
\def\vm{{\bm{m}}}
\def\vn{{\bm{n}}}
\def\vo{{\bm{o}}}
\def\vp{{\bm{p}}}
\def\vq{{\bm{q}}}
\def\vr{{\bm{r}}}
\def\vs{{\bm{s}}}
\def\vt{{\bm{t}}}
\def\vu{{\bm{u}}}
\def\vv{{\bm{v}}}
\def\vw{{\bm{w}}}
\def\vx{{\bm{x}}}
\def\vy{{\bm{y}}}
\def\vz{{\bm{z}}}

% Elements of vectors
\def\evalpha{{\alpha}}
\def\evbeta{{\beta}}
\def\evepsilon{{\epsilon}}
\def\evlambda{{\lambda}}
\def\evomega{{\omega}}
\def\evmu{{\mu}}
\def\evpsi{{\psi}}
\def\evsigma{{\sigma}}
\def\evtheta{{\theta}}
\def\eva{{a}}
\def\evb{{b}}
\def\evc{{c}}
\def\evd{{d}}
\def\eve{{e}}
\def\evf{{f}}
\def\evg{{g}}
\def\evh{{h}}
\def\evi{{i}}
\def\evj{{j}}
\def\evk{{k}}
\def\evl{{l}}
\def\evm{{m}}
\def\evn{{n}}
\def\evo{{o}}
\def\evp{{p}}
\def\evq{{q}}
\def\evr{{r}}
\def\evs{{s}}
\def\evt{{t}}
\def\evu{{u}}
\def\evv{{v}}
\def\evw{{w}}
\def\evx{{x}}
\def\evy{{y}}
\def\evz{{z}}

% Matrix
\def\mA{{\bm{A}}}
\def\mB{{\bm{B}}}
\def\mC{{\bm{C}}}
\def\mD{{\bm{D}}}
\def\mE{{\bm{E}}}
\def\mF{{\bm{F}}}
\def\mG{{\bm{G}}}
\def\mH{{\bm{H}}}
\def\mI{{\bm{I}}}
\def\mJ{{\bm{J}}}
\def\mK{{\bm{K}}}
\def\mL{{\bm{L}}}
\def\mM{{\bm{M}}}
\def\mN{{\bm{N}}}
\def\mO{{\bm{O}}}
\def\mP{{\bm{P}}}
\def\mQ{{\bm{Q}}}
\def\mR{{\bm{R}}}
\def\mS{{\bm{S}}}
\def\mT{{\bm{T}}}
\def\mU{{\bm{U}}}
\def\mV{{\bm{V}}}
\def\mW{{\bm{W}}}
\def\mX{{\bm{X}}}
\def\mY{{\bm{Y}}}
\def\mZ{{\bm{Z}}}
\def\mBeta{{\bm{\beta}}}
\def\mPhi{{\bm{\Phi}}}
\def\mLambda{{\bm{\Lambda}}}
\def\mSigma{{\bm{\Sigma}}}

% Tensor
\DeclareMathAlphabet{\mathsfit}{\encodingdefault}{\sfdefault}{m}{sl}
\SetMathAlphabet{\mathsfit}{bold}{\encodingdefault}{\sfdefault}{bx}{n}
\newcommand{\tens}[1]{\bm{\mathsfit{#1}}}
\def\tA{{\tens{A}}}
\def\tB{{\tens{B}}}
\def\tC{{\tens{C}}}
\def\tD{{\tens{D}}}
\def\tE{{\tens{E}}}
\def\tF{{\tens{F}}}
\def\tG{{\tens{G}}}
\def\tH{{\tens{H}}}
\def\tI{{\tens{I}}}
\def\tJ{{\tens{J}}}
\def\tK{{\tens{K}}}
\def\tL{{\tens{L}}}
\def\tM{{\tens{M}}}
\def\tN{{\tens{N}}}
\def\tO{{\tens{O}}}
\def\tP{{\tens{P}}}
\def\tQ{{\tens{Q}}}
\def\tR{{\tens{R}}}
\def\tS{{\tens{S}}}
\def\tT{{\tens{T}}}
\def\tU{{\tens{U}}}
\def\tV{{\tens{V}}}
\def\tW{{\tens{W}}}
\def\tX{{\tens{X}}}
\def\tY{{\tens{Y}}}
\def\tZ{{\tens{Z}}}


% Graph
\def\gA{{\mathcal{A}}}
\def\gB{{\mathcal{B}}}
\def\gC{{\mathcal{C}}}
\def\gD{{\mathcal{D}}}
\def\gE{{\mathcal{E}}}
\def\gF{{\mathcal{F}}}
\def\gG{{\mathcal{G}}}
\def\gH{{\mathcal{H}}}
\def\gI{{\mathcal{I}}}
\def\gJ{{\mathcal{J}}}
\def\gK{{\mathcal{K}}}
\def\gL{{\mathcal{L}}}
\def\gM{{\mathcal{M}}}
\def\gN{{\mathcal{N}}}
\def\gO{{\mathcal{O}}}
\def\gP{{\mathcal{P}}}
\def\gQ{{\mathcal{Q}}}
\def\gR{{\mathcal{R}}}
\def\gS{{\mathcal{S}}}
\def\gT{{\mathcal{T}}}
\def\gU{{\mathcal{U}}}
\def\gV{{\mathcal{V}}}
\def\gW{{\mathcal{W}}}
\def\gX{{\mathcal{X}}}
\def\gY{{\mathcal{Y}}}
\def\gZ{{\mathcal{Z}}}

% Sets
\def\sA{{\mathbb{A}}}
\def\sB{{\mathbb{B}}}
\def\sC{{\mathbb{C}}}
\def\sD{{\mathbb{D}}}
% Don't use a set called E, because this would be the same as our symbol
% for expectation.
\def\sF{{\mathbb{F}}}
\def\sG{{\mathbb{G}}}
\def\sH{{\mathbb{H}}}
\def\sI{{\mathbb{I}}}
\def\sJ{{\mathbb{J}}}
\def\sK{{\mathbb{K}}}
\def\sL{{\mathbb{L}}}
\def\sM{{\mathbb{M}}}
\def\sN{{\mathbb{N}}}
\def\sO{{\mathbb{O}}}
\def\sP{{\mathbb{P}}}
\def\sQ{{\mathbb{Q}}}
\def\sR{{\mathbb{R}}}
\def\sS{{\mathbb{S}}}
\def\sT{{\mathbb{T}}}
\def\sU{{\mathbb{U}}}
\def\sV{{\mathbb{V}}}
\def\sW{{\mathbb{W}}}
\def\sX{{\mathbb{X}}}
\def\sY{{\mathbb{Y}}}
\def\sZ{{\mathbb{Z}}}

% Entries of a matrix
\def\emLambda{{\Lambda}}
\def\emA{{A}}
\def\emB{{B}}
\def\emC{{C}}
\def\emD{{D}}
\def\emE{{E}}
\def\emF{{F}}
\def\emG{{G}}
\def\emH{{H}}
\def\emI{{I}}
\def\emJ{{J}}
\def\emK{{K}}
\def\emL{{L}}
\def\emM{{M}}
\def\emN{{N}}
\def\emO{{O}}
\def\emP{{P}}
\def\emQ{{Q}}
\def\emR{{R}}
\def\emS{{S}}
\def\emT{{T}}
\def\emU{{U}}
\def\emV{{V}}
\def\emW{{W}}
\def\emX{{X}}
\def\emY{{Y}}
\def\emZ{{Z}}
\def\emSigma{{\Sigma}}

% entries of a tensor
% Same font as tensor, without \bm wrapper
\newcommand{\etens}[1]{\mathsfit{#1}}
\def\etLambda{{\etens{\Lambda}}}
\def\etA{{\etens{A}}}
\def\etB{{\etens{B}}}
\def\etC{{\etens{C}}}
\def\etD{{\etens{D}}}
\def\etE{{\etens{E}}}
\def\etF{{\etens{F}}}
\def\etG{{\etens{G}}}
\def\etH{{\etens{H}}}
\def\etI{{\etens{I}}}
\def\etJ{{\etens{J}}}
\def\etK{{\etens{K}}}
\def\etL{{\etens{L}}}
\def\etM{{\etens{M}}}
\def\etN{{\etens{N}}}
\def\etO{{\etens{O}}}
\def\etP{{\etens{P}}}
\def\etQ{{\etens{Q}}}
\def\etR{{\etens{R}}}
\def\etS{{\etens{S}}}
\def\etT{{\etens{T}}}
\def\etU{{\etens{U}}}
\def\etV{{\etens{V}}}
\def\etW{{\etens{W}}}
\def\etX{{\etens{X}}}
\def\etY{{\etens{Y}}}
\def\etZ{{\etens{Z}}}

% The true underlying data generating distribution
\newcommand{\pdata}{p_{\rm{data}}}
% The empirical distribution defined by the training set
\newcommand{\ptrain}{\hat{p}_{\rm{data}}}
\newcommand{\Ptrain}{\hat{P}_{\rm{data}}}
% The model distribution
\newcommand{\pmodel}{p_{\rm{model}}}
\newcommand{\Pmodel}{P_{\rm{model}}}
\newcommand{\ptildemodel}{\tilde{p}_{\rm{model}}}
% Stochastic autoencoder distributions
\newcommand{\pencode}{p_{\rm{encoder}}}
\newcommand{\pdecode}{p_{\rm{decoder}}}
\newcommand{\precons}{p_{\rm{reconstruct}}}

\newcommand{\laplace}{\mathrm{Laplace}} % Laplace distribution

\newcommand{\E}{\mathbb{E}}
\newcommand{\Ls}{\mathcal{L}}
\newcommand{\R}{\mathbb{R}}
\newcommand{\emp}{\tilde{p}}
\newcommand{\lr}{\alpha}
\newcommand{\reg}{\lambda}
\newcommand{\rect}{\mathrm{rectifier}}
\newcommand{\softmax}{\mathrm{softmax}}
\newcommand{\sigmoid}{\sigma}
\newcommand{\softplus}{\zeta}
\newcommand{\KL}{D_{\mathrm{KL}}}
\newcommand{\Var}{\mathrm{Var}}
\newcommand{\standarderror}{\mathrm{SE}}
\newcommand{\Cov}{\mathrm{Cov}}
% Wolfram Mathworld says $L^2$ is for function spaces and $\ell^2$ is for vectors
% But then they seem to use $L^2$ for vectors throughout the site, and so does
% wikipedia.
\newcommand{\normlzero}{L^0}
\newcommand{\normlone}{L^1}
\newcommand{\normltwo}{L^2}
\newcommand{\normlp}{L^p}
\newcommand{\normmax}{L^\infty}

\newcommand{\parents}{Pa} % See usage in notation.tex. Chosen to match Daphne's book.

%\DeclareMathOperator*{\argmax}{arg\,max}
%\DeclareMathOperator*{\argmin}{arg\,min}

%\DeclareMathOperator{\sign}{sign}
%\DeclareMathOperator{\Tr}{Tr}
\let\ab\allowbreak

% to compile a preprint version, e.g., for submission to arXiv, add add the
% [preprint] option:
%     \usepackage[preprint]{ofu_xai_2022}


% to compile a camera-ready version, add the [final] option, e.g.:
% \usepackage[final, nonatbib]{ofu_xai_2022}	

% to avoid loading the natbib package, add option nonatbib:
%    \usepackage[nonatbib]{ofu_xai_2022}


\usepackage[utf8]{inputenc} % allow utf-8 input
\usepackage[T1]{fontenc}    % use 8-bit T1 fonts
\usepackage{hyperref}       % hyperlinks
\usepackage{url}            % simple URL typesetting
\usepackage{booktabs}       % professional-quality tables
\usepackage{amsfonts}       % blackboard math symbols
\usepackage{nicefrac}       % compact symbols for 1/2, etc.
\usepackage{microtype}      % microtypography
\usepackage{xcolor}         % colors
\usepackage{wrapfig}        % wrap figure in page
\usepackage{graphicx}       % for includegraphics
\usepackage{caption}        % Add figure captions
\usepackage{multicol}       % Add multiple columns
\usepackage[most]{tcolorbox} % Add boundaries to images
\usepackage[export]{adjustbox} % Change alignment of figures


% \usepackage[round]{natbib}
%%%%

\title{Seminar on xAI application in detection and survival rate prediction of breast cancer patients}


% The \author macro works with any number of authors. There are two commands
% used to separate the names and addresses of multiple authors: \And and \AND.
%
% Using \And between authors leaves it to LaTeX to determine where to break the
% lines. Using \AND forces a line break at that point. So, if LaTeX puts 3 of 4
% authors names on the first line, and the last on the second line, try using
% \AND instead of \And before the third author name.


\author{%
  Sridhara Madhu Mohan Nelemane\thanks{Alternate email: snmohan83@gmail.com; github: https://github.com/snmohan83/xAI} \\
  Otto-Friedrich University of Bamberg\\
  96049 Bamberg, Germany\\
  \texttt{sridhara-madhu-mohan.nelemane@stud.uni-bamberg.de} \\[0.5cm]
  Seminar: xAI-Sem-Summer-2022 \\
  Degree: M.Sc. ISoSySc \\
  Matriculation \#: 1941196
  % examples of more authors
  % \And
  % Coauthor \\
  % Affiliation \\
  % Address \\
  % \texttt{email} \\
  % \AND
  % Coauthor \\
  % Affiliation \\
  % Address \\
  % \texttt{email} \\
  % \And
  % Coauthor \\
  % Affiliation \\
  % Address \\
  % \texttt{email} \\
  % \And
  % Coauthor \\
  % Affiliation \\
  % Address \\
  % \texttt{email} \\
}


\begin{document}


\maketitle
\def\va{{\bm{a}}}

\begin{abstract}
  12.5\% of all cancers in the world are breast cancers. It is also the most common cause of \textit{"death due to illness"} among women. Owing to the commonality of the cancer, we also have availability of significant amount of data from previous patients. Several research efforts are undertaken both in the medical and technology communities to advance detection and treatment effectiveness for cancer. Traditional methods that use human readers is time-consuming and also suffer from short supply of medical professionals. The available Machine Learning tend to black-box its learning stages and the biases thereby losing trust among medical decision makers and patients. Explainable AI methods therefore provide a solution to both these problems. xAI provides solution in three dimensions: 1) It saves effort from medical professionals by aiding them in reading mammograms, scan reports and slide images. 2) It provides novel insghts into understanding the causes of cancer and devicing better treatment solutions. 3) It can provide early-detection by analysing pre-conditions that are not obvious to the human eye. Through this seminar report, we examine all of the 3 dimensions in which xAI can improve care for breast cancer patients. The report also describes the challenges and future work in this aspect.
\end{abstract}


\section{Breast Cancer Clinical Detection Methods}

For the purpose of this report, 5 methods of clinically detecting breast cancer are considered and explained below:

\begin{enumerate}
\item \textbf{Physical Examination} - Done by a medical professional or a family doctor and normally detection is possible only when a lump can be physically felt. When detected in such a stage, the cancer is already developed and could be an advanced stage thereby not really helpful in controlling the disease.

\item \textbf{Mammography} - This is a technique similar to X-ray but focussed on the breast tissues. The result of a Mammography is a digital image that can be conviniently read by a medical professional or fed into a digital system for analysis. The reading of Mammograms is however time-consuming and also prone to human errors. Therefore, AI method have been deviced to provide higher accuracy and better sensitivity. This report provides analysis of such techniques.

\item \textbf{Digital Pathology} - This technique use whole slide images of breast tissue samples for analysis under a microscope. Digital techniques are preferred to human reading of these images due to the effort needed, shortage of medical professionals time and also better accuracy.

\item \textbf{Deep Tissue Biopsy} - This is a confirmation technique done when a positive result is shown in one of the other methods like Mammography. This involves a minor surgical procedure to extract a sample of the breast tissue from the suspected region and detailed chemical and microscopic analysis to determine the type and characteristics of the cancer. It is rather used as a technique for classification of cancer that helps in determining the course of treatment. Since the procedure requires a professional surgeon to perform the incision and a standard Operation Theatre arrangement, it is an expensive method used only for confirmation.

\item \textbf{PET Scan} - Positron Emission Tomography produces a 3-dimensional detailed image of the body and is helpful in identifying the degree of metastasis of cancer to breast and other parts of the body. This is key in identifying the extent of surgical procedure and type of therapies to be used for treatment. However, this is an expensive test and requires high-end equipment not reachable to most less-privileged populations.

\end{enumerate}

As stated earlier, this report focusses on AI/xAI techniques applied on Mammography images which is easy to obtain through mass-screening efforts. The availability of diverse data helps in developing effective AI models with explainability.

\section{Mass Screening and Evaluation Efforts}

\subsection{The Challenges}

Designing AI systems for Breast Cancer detection and analysis is faced with the following challenges:

\paragraph{Data availability} Training and developing an effective model requires huge number of Mammography images with various demographic characteristics.

\paragraph{Follow-up} To develop early-detection models, its necessary to follow-up patients that are screened to verify if they develop cancer. The gap between the first screening and the follow-up also needs to be decided. 

\paragraph{Geographical Differences} Even with available data, the screening results from different geographic regions differ. There are also difference across various racial profiles of patients and their environment. Such discrepancies influence the models and therefore the predictions produced by such models

\paragraph{Anonimity and Privacy} Different privacy regulations in different regions prevents obtaining screening data and also follow-up.

Some of the efforts undertaken by various organizations in obtaining useful mammography data and application of AI methods for cancer detection are listed below.
\begin{itemize}
\item CAMELYON Challenge (periodically) - Challenges to invent accurate prediction models for cancer detection from images
\item Dream2016 Digital Mammography - Mass screening for Mammography data
\item PASCAL VOC 2009 - Anonymous mammography images
\end{itemize}

\subsection{Evaluation Data Sets}
The AI system evaluation presented in \cite{Nature577} analyses screening mammography image datasets from patiensts in the US and the UK. Table \ref{screening} is a comparitive table of the datasets from the two regions.

\begin{table}
  \caption{Breast cancer screening datasets - US and UK (Source: Nature577:89-94,2020)}
  \label{screening}
  \centering
  \begin{tabular}{lll}
    \toprule
    Parameter        &  UK       &     US  \\
    \midrule
    Dataset Size     & 25856     &    3097     \\
    Biopsied         & 785       &    1511    \\
    Follow-up        & 39 months &    27 months  \\
    Cancer Positive  & 414       &    686  \\
    Sensitivity over human readers & 2.7\% & 9.4\%  \\
    Specificity over human readers & 1.8\% & 5.7\%  \\
    \bottomrule
  \end{tabular}
\end{table}

\subsection{The Ground truth determination and Evaluation}

\begin{itemize}
\item Positive if biopsy-confirmed within 3 months from after the screening interval
\item Otherwise negative if a second exam occured before the screening interval
\item Compare performance against clinician read results
\item Generalization of data sets by training on UK Training set and testing on US test set
\item 6 radiologists read 500 cases from US test set independent of each other. The clinicians assign a BI-RADs score to determine the effectiveness of the reading.
\item The AI system used three deep learning models analysing at various levels producing a cance-risk score of 0 to 1.
\end{itemize}

The figure \ref{aieval} shows the sensitivity and specifity improvements of AI system over traditional systems. 

\begin{figure} [ht]
  \centering
  \tcbox{\includegraphics[width=0.95\textwidth]{images/aisystemparams.png}}
    \caption{Evaluation of detection results from Breast Cancer screening}
    \caption*{Source: Nature 577:89-94. 2020}
  \label{aieval}
\end{figure}


\subsection{Results from screening and evaluation}

When mammography images are available in a considerable number along with the follow-up results, it could be possible to develop and train effective AI models. These models show significant improvement in detection compared to manual effort and can be applied across geographies. The AI system significantly reduces false positives and false-negatives. During the evaluation, there were few cases that were missed by the AI system and detected by clinician reading and also found were cases which were missed by clinicians but discovered by the AI system. However, the evaluation reduced the effort by second readers by upto 88\% which is a significant contribution owing to shortage of clinical professionals. This evaluation provides an evidence that improvements in AI systems in Breast Cancer Detection and Analysis is useful in making cancer cures more accessible and hence encourages further research and development.

The predictions produced in the described and other AI systems are based on black-box deep neural network models with several layers. In order to fully adapt AI methods in clinical processes, the clinicians demand clear explanations on the reasons for prediction. Also, along with prediction, there is also a need for classification and staging of cancer for effective treatments. This can be achieved using Explainable AI systems. The section \ref{xaixgboost} presents a technique derived from backpropagation to help identify and classify the micro-environment that significantly contribute to cancer development and the section \ref{xaiductal} provides a method to help in early-detection(stage 0) of cancer from mammographs using a well-known mathematical technique.


\section{Explainable AI to study Tumor-Micro Environment}
\label{xaixgboost}

\subsection{Background}

An important aspect of effective cancer care for breast cancer patients is to identify the type of cancer and the biological factors that influence the development of cancers. Particularly there are 4 types of breast cancers:
\begin{enumerate}
\item Estrogen Receptors (ER)
\item Progesterone Receptors (PR)
\item Human Epidermal growth factor Receptor 2(HER2)
\item Triple Negative Breast Cancer (TNBC) ( a combination of all the three above)
\end{enumerate}

The cells living in the breast and surrounding tissues like T,B Lymphocytes, CD4+T cells, CD8+T cells, and several harmones that influence the development of breast cancer constitute the Tumor Micro Environment (TME). Along with the type of cancer, it is also important to understand the TMEs that contribute to cancer development. This will help identify the risk of cancer and also survivability of patients past treatment at an early stage. 
\cite{cancers2021} presents an xAI method that helps identify the TME factors that has significant contributions and thereby giving new insights into pre-detection of cancer and therefore allow for precautionary and preventive measures for patients with higher risks. The technique described in \ref{xgboost} uses datasets from http://www.cbioportal.org/ and https://gdac.broadinstitute.org/ .

\subsection{XGBoost method - The experiment and result}
\label{xgboost}
XGBoost is a tree-based search algorithm that searches for the best outcome. The outcome is defined to be 5-year survival rate which is given by:

\begin{equation}
\hat{y} = \phi(x) = \frac{1}{n}\sum_{k=1}^{n}f_k(x)
\end{equation}

Where, \(\hat{y}\) is the predicted outcome - overall survival and 5-year survival (\(1\leq{k}\leq{n}\)) and \(f_1, f_2, \ldots, f_n\) are functions learned by \(n\)
(\cite{cancers2021})

The learning functions aim at minimizing the value of a regularized objective based on the gap between the outcome and predicted outcome given by:

\begin{equation}
\mathcal{L}(\phi) = \sum_{i} l(\hat{y} - y) + \sum_{k} \Omega(f_k)
\end{equation}

Where, an extra regularization term \(\Omega\) is used to a penalize tree overgrowth and avoid overfitting given by:

$$ \Omega(f_k) = \gamma T + \frac{1}{2} \lambda ||w||^2 $$

And,

\(l\) - is the differentiable convex loss function – measures the difference between predictions\\
\(T\) - is the number of leaves in a tree\\
\(\lambda\) - is the penalty parameter\\
\(||w||\) - is the vector of scores on the leaves\\
\(\gamma\) - is the complexity of the leaf\\

Without the \(\Omega\) term, the equation(2) returns the standard gradient term that looks like:

\begin{equation}
\mathcal{L}(\phi) = \sum_{i} l(\hat{y} - y)
\end{equation}


\paragraph{SHAP} is a game-theory based approach used to model the contribution of each feature towards the final outcome and is determined by the Shapely values assigned to each feature. In effect, SHAP computes the average marginal contribution of each feature across all possible combinations of features. The larger the Shapely value, the greater the impact of the feature on the 5-year survivability of the patient. The plot in figure \ref{xgboostres} sorts the feature in the decreasing order of Shapely values.

\begin{figure}
  \centering
  \tcbox{\includegraphics[scale=0.25]{images/xgboostresults.png}}
    \caption{XGBoost results in the decreasing order of shapely values for each TME feature}
    \caption*{Source: Chakraborty et.al, Cancers 2021, 13, 3450.}
  \label{xgboostres}
\end{figure}

\paragraph{Insight} Observing the Shapely values in figure \ref{xgboostres}, the experiment concluded that B cells, M0 Macrophages, CD8+ T Cells, and NK T Cells play significant role in determining survivability of patients. A precise contribution threshold values were also determined for each type of TME as shown in figure \ref{xgbooststat} .

\begin{figure}
  \centering
  \tcbox{\includegraphics[scale=0.25]{images/shapresults.png}}
    \caption{Contribution by each cell based on Shapely values}
    \caption*{Source: Chakraborty et.al, Cancers 2021, 13, 3450.}
  \label{xgbooststat}
\end{figure}

In conclusion, xAI methods simlar to the one shown here can also be used in improving the knowledge around the causes and spread of cancer which help in prognosis and treatment.


\section{Early Detection through xAI Techniques}
\label{xaiductal}

\subsection{Introducing Ductal Carcinoma}

\textbf{Ductal Carcinoma In Situ} refers to Stage-0 breast cancer which begins in the ductal area of the breast and has yet no signs of \textit{metastasis} (\textit{metastasis} refers to the extent of spread of cancer in the body). The drawing in figure \ref{ductal} shows the inflection point of such a cancer growth ( \cite{dcis2021} ). 

\begin{figure} [!ht]
  \centering
  \includegraphics[scale=0.35]{images/ductal1.png}
    \caption{Development of Ductal Carcinoma in Situ}
    \caption*{Source: https://www.cancer.org/cancer/breast-cancer/about/types-of-breast-cancer/dcis.html}
  \label{ductal}
\end{figure}


A stage-0 cancer can be treated with minimal risk and high survivability with simple surgical procedure followed by medication. The ghastly effects of Chemotherapy, Immunotherapy and Radiotherapy treatments employed at later stages can be avoided. However, detection of cancer at search early stage is challenging and particularly so if it is done through mammography and not by using PET scan or biopsy which are expensive.

The paper \cite{insitu2021} presents an xAI approach to detect cancer at such a nascent stage.

\subsection{The method}

The technique uses a method of backpropagation called Layerwise Relevance Propagation which redistributes the input image at each layer backwards. The redistribution uses the formula:

\begin{equation}
\R_{j} = \sum_{k} \frac{x_j w_{j,k}}{\sum_k x_j w_{j,k} + \epsilon} \R_{k}
\end{equation}

The application of relevance propogation algorithm is shown in figure \ref{relevance}

\begin{figure} [ht]
  \centering
  \includegraphics[scale=0.5]{images/lrpalgorithm.png}
    \caption{Layerwise Relevance Propagation Algorithm}
    \caption*{Source: https://danshiebler.com/2017-04-16-deep-taylor-lrp/}
  \label{relevance}
\end{figure}

A suitable vector-valued relevance propagation function should be defined that can backpropagate to the input to produce a heat map around the most significant features that contributed to the outcome at each layer. \cite{bach2015} lists several constraints on such a relevance function

However, \cite{insitu2021} uses \textbf{Deep Taylor Decomposition} as a relevance funcion. This is based on the Taylor series expansion that approximates the value of a function around a root point and to ease the process, sets the root point to 0 resulting in the simplified Taylor series as shown in figure \ref{taylor1} and \ref{taylor2}. The blog \cite{schiebler17} provides a simpler explanation of the Deep Taylor approach to Layerwise Propagation technique.
\begin{figure} [ht]
  \centering
  \includegraphics[scale=0.25]{images/dtlrpeqn1.png}
    \caption{Original Taylor Series Expansion}
    \caption*{Source: https://danshiebler.com/2017-04-16-deep-taylor-lrp/}
  \label{taylor1}
\end{figure}
\begin{figure} [ht]
  \centering
  \includegraphics[scale=0.25]{images/dtlrpeqn2.png}
    \caption{Approximated Taylor Series Expansion with root point set to 0}
    \caption*{Source: https://danshiebler.com/2017-04-16-deep-taylor-lrp/}
  \label{taylor2}
\end{figure}

The root points are located around the edges of decision points and therefore the gradient around these points give maximum information about the classification and give better approximation (explanation from \cite{schiebler17}).

The application of Deep Taylor Approximation for Layerwise Relevance Propagation is shown in figure \ref{dtlrptech} .

\begin{figure} [ht]
  \centering
  \tcbox{\includegraphics[scale=0.35]{images/dtlrptechnique.png}}
    \caption{Deep Taylor Decomposition with LRP}
    \caption*{Source: M. La Ferla et. al arxiv eess.IV 2021}
  \label{dtlrptech}
\end{figure}

\clearpage
\subsection{Results in Breast Cancer Detection}

\cite{insitu2021} shows the result of applying Deep Taylor Decomposition on Layerwise Relevance Propagation on breast cancer patients. It classifies the images into 3 types based on the presence of cancer cells as Normal, Benign and Malignant. The presence of cancer cells is marked in a different color on the heatmap and this was possible through relevance propogation. 
The input images were used with Resnet-50 patch classifier and the explanation was provided using the Deep Taylor LRP method in the form of heat map images. 

The evaluation presented in \cite{insitu2021} says that Deep Taylor LRP can be applied on CNN for providing clinically relevant explanations for images classfied by ReLU activation functions. However, the performance is not satisfactory. But, they also provide evidence that other methods are less performant compared to Deep Taylor LRP and therefore recommend further research into the area.


\section{Conclusion and Remarks}

Breast cancer being the most dominant of all terminal diseases among women, calls for improved diagnosis, prognosis and treatment methods. Traditional methods require investment in equipment like PET scan and more clinical professionals as manual reading of images from mammograph or slide images are time-consuming and prone to error. AI methods offer an alternative to improve breast cancer care in several aspects. The following conclusions can be derived: 

\begin{enumerate}
\item Mass screening can help develop accurate well-trained AI models which can be used by AI system to analyze new images thereby reducing the load on medical professionals. 
\item Achieve improved accuracy due to lesser human error and quicker results
\item XAI methods on huge datasets can provide insights into the TME features that cause cancer and also provide anlaysis on survival rates of patients
\item XAI methods with more research and improvements can be employed in clinical trials for detecting cancer at very early stages enabling preventive measures for patients involving risks.
\item The xAI models particularly the relevance models for explaining In Situ cancer are not as performant as they need to be and advanced research into other possible relevance functions and alternative methods of explainable AI to improve confidence among clinical professionals.
\end{enumerate}

\clearpage
\section{Declaration of Authorship}

\begin{figure} [ht]
  \includegraphics[width=1.2\textwidth,left]{images/authorship.png}
\end{figure}

\clearpage
\bibliography{bibliography}
\bibliographystyle{abbrvnat}

\end{document}
